\section{Tabla Periódica}
    \subsection{Electronegatividad}
    Se define como la capacidad de atraer electrones; esta propiedad está estrechamente relacionado con la estructura electrónica del elemento. Por ejemplo, 

    % \subsection*{Version 2 -- \pkg{chemfig} and \pkg{elements}}

    % \setcharge{debug}
% \Charge{30=\:,120:3pt=$\ominus$,210:5pt=$\delta^+$}{Fe}\qquad
% \Charge{[circle]30=\:,
% 120[circle,anchor=180+\chargeangle]=$\ominus$,
% 210[anchor=180+\chargeangle]=$\delta^+$}{Fe}





\schemestart
    \chemname[1.5ex]{\charge{180=\eSingleDot}{Li}}{\scriptsize\elconf{Li}}
    \+{1.5em}
    \chemname[1.5ex]{
        \charge{
            0=\eSingleDot,
            90=\ePairDot,
            180=\ePairDot,
            270=\ePairDot
        }{F}
    }{\scriptsize\elconf{F}}
    \arrow{->}
    \chemname[1.5ex]{\charge{45=\pChargeLewis{}}{Li}}{\scriptsize\writeelconf{2}}
    \+{1em}
    \chemname[1.5ex]{
        \charge{
            0=\ePairDot,
            90=\ePairDot,
            180=\ePairDot,
            270=\ePairDot,
            45=\nChargeLewis{}
        }{F}
    }{\scriptsize\writeelconf{2,2+6}}
\schemestop




    $\text{metales} \ll \ce{Te, P, H, As, B, Si} < \ce{I, S, C, Se} < \ce{Br} < \ce{Cl, N} < \ce{O} < \ce{F}$
