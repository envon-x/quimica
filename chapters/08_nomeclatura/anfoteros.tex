\section*{ANFÓTEROS}
Los elementos anfóteros forman los óxidos anfóteros los cuales son capaces de neutralizar los ácidos y bases; es decir, con ciertos estados de oxidación altos, el metal forma \textit{enlaces más covalentes} y sus óxidos muestran comportamiento ácido-base intermedio; esto de debe a la alta densidad de carga del catión, a una mayor polarización del enlace $\ce{M-O}$ y a un menor carácter iónico.

\ch{Al(OH)3 <=> Al^3+ + 3 \ohColor{OH^-}}
Aquí el \ch{Al(OH)3} se comporta como una \textbf{base} porque genera \ch{OH-}

\ch{Al(OH)3 <=> AlO3 H2^{3+} + \hColor{H^{+}}}
Aquí el \ch{Al(OH)3} se comporta como un \textbf{ácido} porque genera \ch{H+}

\begin{tabular}{|>{\centering\arraybackslash}m{6cm}|m{8cm}|}
\hline
\textbf{Reacción de formación} & \textbf{Nomenclaturas} \\
\hline

\reaccionNomenclatura
  {4Al + 3O2 -> 2Al2O3}
  {óxido alumínico}
  {óxido de aluminio (III)}
  {trióxido de dialuminio}
\\
\hline

\reaccionNomenclatura
  {2Zn + O2 -> 2ZnO}
  {óxido zincoso}
  {óxido de zinc (II)}
  {monóxido de zinc}
\\
\hline

\reaccionNomenclatura
  {2Zn + O2 -> 2ZnO}
  {óxido zincoso}
  {óxido de zinc (II)}
  {monóxido de zinc}
\\
\hline
\end{tabular}