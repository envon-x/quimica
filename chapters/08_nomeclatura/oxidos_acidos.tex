\section*{ÓXIDO ÁCIDOS O ANHIDRIDOS}

\large
Son compuestos formados por un \textbf{NO METAL}($ \ce{Cl}, \ce{Br}, \ce{I}, \ce{S}, \ce{Se}, \ce{N}, \ce{P}, \ce{Si} $) y \textbf{OXÍGENO}

% \vspace{0.8cm}


\begin{center}
    \colorbox{orange!20}{ % Cambia 'yellow!20' por el color que prefieras
        \parbox{0.8\textwidth}{
        \centering
        \Large
        \textbf{NO METAL + OXÍGENO $ \longrightarrow $ ANHÍDRIDO}
        
        \vspace{0.3cm}
        \begin{flushleft} % Nota: flushleft en minúsculas es el estándar
            Es decir,
        \end{flushleft}
    
        \textbf{NM$^{+y}$ + O$^{-2}$ $ \longrightarrow $ {NM}$_2$O$_y$}
        }
    }
\end{center}

Algunos metales no se comportan siempre como tales, cuando tienen un estado de oxidación (valencia) muy alto, sus óxidos dejan de ser básicos y se vuelven ácidos; es decir, con las valencias más altas se comporta como no metales. %\textit{no metales}.\\

\begin{flushleft}
    
    \textbf{Manganeso}($\ce{Mn}$) con \eoExp ($+6$, $+7$), forman anhidridos u óxidos ácidos $ \ce{MnO3}$, $\ce{Mn2O7}$.\\
    \textbf{Cromo}($\ce{Cr}$) con \eoExp ($+6$), forman $\ce{CrO3}$\\
    \textbf{Vanadio}($\ce{V}$) con \eoExp (+5) forma $\ce{V2O5}$ tiene un comportamiento predominantemente ácido.
\end{flushleft}


\vspace{1cm}

Para nombrar estos compuestos se tiene las siguientes reglas:

\vspace{0.5cm}

% Diagrama con llaves usando TikZ
\begin{center}
    \begin{tikzpicture}[scale=1.2]

        % Cuadro izquierdo: 4 valencias
        \node[draw, rectangle, thick, minimum width=2.0cm, minimum height=1.0cm, align=center] (v4) at (-0.1,0) {Si E\\tiene \textbf{4}\\valencias};

        % Opciones para 4 valencias
        \node[draw, rectangle, thick, fill=hipo!50,
          minimum width=2.5cm, minimum height=0.7cm, align=center,
          rounded corners=3pt] (opt4) at (4.5, 1.5) {HIPO - E - OSO};
        \node[draw, rectangle, thick, fill=oso!40,
          minimum width=0.0cm, minimum height=0.7cm, align=center,
          rounded corners=3pt] (opt4b) at (4.5, 0.5) {E - OSO};
        \node[draw, rectangle, thick, fill=ico!40,
          minimum width=0.0cm, minimum height=0.7cm, align=center,
          rounded corners=3pt] (opt4c) at (4.5, -0.5) {E - ICO};
        \node[draw, rectangle, thick, fill=per!50,
          minimum width=1.5cm, minimum height=0.7cm, align=center,
          rounded corners=3pt] (opt4d) at (4.5, -1.5) {PER - E - ICO};

        % Llave para 4 valencias
        \draw[thick, decorate, decoration={brace, amplitude=10pt}] (1.2, -2) -- (1.2, 2);

        % Cuadro central: 1 valencia 
        \node[draw, rectangle, thick, minimum width=2.0cm, minimum height=1.0cm, align=center] (v2) at (2.3, -0.5) {Si E tiene\\\textbf{1} valencia};

        % Llave para una 1 valencia
        \draw[thick, decorate, decoration={brace, amplitude=4pt}] (3.5, -0.9) -- (3.5, -0.1);


        % Cuadro central: 2 valencias  
        \node[draw, rectangle, thick, minimum width=2.5cm, minimum height=1.6cm, align=center] (v2) at (7.2, 0.0) {Los que tienen\\\textbf{2} valencias};

        % Llave interna para 2 valencias
        \draw[thick, decorate, decoration={brace, amplitude=6pt, mirror}] (5.5, -0.8) -- (5.5, 0.8);

        % Cuadro derecho arriba: 3 valencias
        \node[draw, rectangle, thick, minimum width=2.2cm, minimum height=0.8cm, align=center] (v3) at (10.1, 0.5) {Los que\\tienen \textbf{3}\\valencias};


        % Llave para 3 valencias
        \draw[thick, decorate, decoration={brace, amplitude=8pt, mirror}] (8.8, -1.0) -- (8.8, 1.8);

    \end{tikzpicture}
\end{center}

\vspace{0.5cm}

Donde: \textbf{E} = elemento


% Leyenda de colores
\begin{center}
\begin{tikzpicture}
    \node[draw, fill=hipo!50, rounded corners=3pt, minimum width=1.5cm, minimum height=0.5cm] at (0,0) {};
    \node[right] at (0.65,0) {\small Val. más baja};

    \node[draw, fill=oso!40, rounded corners=3pt, minimum width=1.5cm, minimum height=0.5cm] at (4,0) {};
    \node[right] at (4.68,0) {\small Val. menor};

    \node[draw, fill=ico!40, rounded corners=3pt, minimum width=1.5cm, minimum height=0.5cm] at (8,0) {};
    \node[right] at (8.7,0) {\small Val. mayor};

    \node[draw, fill=per!50, rounded corners=3pt, minimum width=1.5cm, minimum height=0.5cm] at (12,0) {};
    \node[right] at (12.7,0) {\small Val. más alta};
\end{tikzpicture}
\end{center}

\vspace{1cm}

\subsection*{Nomenclatura Stock}

\vspace{0.3cm}

\large
{ÓXIDO DE + } \textbf{NO METAL} {(\textbf{VALENCIA del No Metal})}
\vspace{1.5cm}

\subsection*{Nomenclatura IUPAC}

\vspace{0.3cm}

\large
\textbf{PREFIJO + ÓXIDO + DE + PREFIJO + NOMBRE DEL NO METAL}

\vspace{0.5cm}

\normalsize
(Los prefijos indican el número de átomos: mono-, di-, tri-, tetra-, penta-, hexa-, hepta- etc.)




\section*{ÓXIDOS NEUTROS}
También denominados \textbf{anhidridos imprefectos}, estos óxidos no reaccionan con el agua y no se descomponen fácilmente.

% \ch{C \eoTwoPos[]  + C-2 -> CO}
$\ce{{C\eoTwoPos}  +  {O\eoTwoNeg} -> C2O2 -> CO}$

\ch{NO}
\ch{N2O}


