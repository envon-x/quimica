\section{Reacciones químicas y su clasificación}

    %=====================================================
    \subsection{Por su mecanismo}

        \subsubsection*{Reacciones de síntesis o combinación}
        Son aquellas reacciones donde dos o más sustancias se combinan para formar un único compuesto.

        Forma general:
        \[
        \ce{A + B -> AB}
        \]
        
        Ejemplo:
        \[
        \ce{2H2 + O2 -> 2H2O}
        \]

        \subsubsection*{Reacciones de descomposición o análisis}

        Forma general:
        \[
        \ce{AB -> A + B}
        \]

        Ejemplo:
        \[
        \ce{2H2O2 -> 2H2O + O2}
        \]
        \subsubsection*{Reacciones de desplazamiento simple o sustitución}

        \[
        \ce{Zn + 2HCl -> ZnCl2 + H2}
        \]

        Forma general:
        \[
        \ce{A + BC -> AC + B}
        \]

        \subsubsection*{Reacciones de doble desplazamiento o metátesis}
        Forma general:
        \[
        \ce{AB + CD -> AD + CB}
        \]

        Ejemplo:
        \[
        \ce{AgNO3 + NaCl -> AgCl v + NaNO3}
        \]


        %=====================================================
        \subsection{Por su naturaleza}

        \subsubsection*{Reacciones de combustión}

        \[
        \ce{CH4 + 2O2 -> CO2 + 2H2O}
        \]

        \subsubsection*{Reacciones de neutralización o ácido--base}

        \[
        \ce{HCl + NaOH -> NaCl + H2O}
        \]

        \subsubsection*{Reacciones de precipitación}

        \[
        \ce{BaCl2 + Na2SO4 -> BaSO4 v + 2NaCl}
        \]

        \subsubsection*{Reacciones de hidrólisis}

        \[
        \ce{NH4Cl + H2O <=> NH4OH + HCl}
        \]

        %=====================================================
        \subsection{Por su reactividad}

        \subsubsection*{Reacciones rápidas}

        Ocurren casi instantáneamente.
        \[
        \ce{HCl + NaOH -> NaCl + H2O}
        \]

        \subsubsection*{Reacciones lentas}

        Ocurren en largos periodos de tiempo.
        \[
        \ce{4Fe + 3O2 -> 2Fe2O3}
        \]

        \subsubsection*{Reacciones explosivas}

        Liberan gran cantidad de energía en poco tiempo.
        \[
        \ce{2H2 + O2 -> 2H2O}
        \]

        %=====================================================
        \subsection{Por el intercambio calorífico}

\subsubsection*{Reacciones exotérmicas}

Liberan calor hacia el entorno.  
La entalpía de reacción es negativa (\(\Delta H < 0\)).

\[
\ce{CH4(g) + 2O2(g) -> CO2(g) + 2H2O(l)}
\qquad
\Delta H^\circ_{\text{comb}} = -890\ \text{kJ mol}^{-1}
\]

---

\subsubsection*{Reacciones endotérmicas}

Absorben calor del entorno.  
La entalpía de reacción es positiva (\(\Delta H > 0\)).

\[
\ce{CaCO3(s) -> CaO(s) + CO2(g)}
\qquad
\Delta H^\circ = +178\ \text{kJ mol}^{-1}
\]

        %=====================================================
        \subsection{Por el cambio de estado de oxidación}

        \subsubsection*{Reacciones de oxidación--reducción (redox)}

        Oxidación:
        \[
        \ce{Fe^{2+} -> Fe^{3+} + e^-}
        \]

        Reducción:
        \[
        \ce{Cu^{2+} + 2e^- -> Cu}
        \]

        \subsubsection*{Reacciones de desproporción}

        Una especie se oxida y se reduce simultáneamente.
        \[
        \ce{2H2O2 -> 2H2O + O2}
        \]

        %=====================================================
        \subsection{Por la fuente de energía}

        \subsubsection*{Reacciones fotoquímicas}

        Iniciadas por radiación luminosa.
        \[
        \ce{2AgCl ->[\text{luz}] 2Ag + Cl2}
        \]

        \subsubsection*{Reacciones térmicas}

        Requieren calentamiento.
        \[
        \ce{CaCO3 ->[\Delta] CaO + CO2}
        \]

        \subsubsection*{Reacciones electroquímicas}

        Ocurren mediante corriente eléctrica.

        Ánodo:
        \[
        \ce{Zn -> Zn^{2+} + 2e^-}
        \]

        Cátodo:
        \[
        \ce{Cu^{2+} + 2e^- -> Cu}
        \]

        %=====================================================
        \subsection{Por su reversibilidad}

        \subsubsection*{Reacciones irreversibles}

        Avanzan en un solo sentido.
        \[
        \ce{CH4 + 2O2 -> CO2 + 2H2O}
        \]

        \subsubsection*{Reacciones reversibles}

        Ocurren en ambos sentidos.
        \[
        \ce{N2 + 3H2 <=> 2NH3}
        \]
