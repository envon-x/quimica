\begin{table}[h]
\centering
\begin{tabular}{|>{\columncolor{metal}}c|c|c|c|}
\hline
\rowcolor{header}
\textcolor{white}{\textbf{Elemento}} & \textcolor{white}{\textbf{Símbolo}} & \textcolor{white}{\textbf{Valencia}} & \textcolor{white}{\textbf{Estado de Oxidación}} \\
\hline
Litio & Li & 1 & +1 \\
\hline
Sodio & Na & 1 & +1 \\
\hline
Potasio & K & 1 & +1 \\
\hline
Rubidio & Rb & 1 & +1 \\
\hline
Cesio & Cs & 1 & +1 \\
\hline
Berilio & Be & 2 & +2 \\
\hline
Magnesio & Mg & 2 & +2 \\
\hline
Calcio & Ca & 2 & +2 \\
\hline
Estroncio & Sr & 2 & +2 \\
\hline
Bario & Ba & 2 & +2 \\
\hline
\end{tabular}
\caption{Metales de valencia única}
\end{table}

\section*{Metales de Transición y Otros Metales}

\begin{longtable}{|>{\columncolor{transicion}}c|c|c|c|}
\hline
\rowcolor{header}
\textcolor{white}{\textbf{Elemento}} & \textcolor{white}{\textbf{Símbolo}} & \textcolor{white}{\textbf{Valencias}} & \textcolor{white}{\textbf{Estados de Oxidación}} \\
\hline
\endfirsthead

\hline
\rowcolor{header}
\textcolor{white}{\textbf{Elemento}} & \textcolor{white}{\textbf{Símbolo}} & \textcolor{white}{\textbf{Valencias}} & \textcolor{white}{\textbf{Estados de Oxidación}} \\
\hline
\endhead

Hierro & Fe & 2, 3 & +2, +3 \\
\hline
Cobre & Cu & 1, 2 & +1, +2 \\
\hline
Plata & Ag & 1 & +1 \\
\hline
Oro & Au & 1, 3 & +1, +3 \\
\hline
Cinc & Zn & 2 & +2 \\
\hline
Mercurio & Hg & 1, 2 & +1, +2 \\
\hline
Cadmio & Cd & 2 & +2 \\
\hline
Aluminio & Al & 3 & +3 \\
\hline
Plomo & Pb & 2, 4 & +2, +4 \\
\hline
Estaño & Sn & 2, 4 & +2, +4 \\
\hline
Níquel & Ni & 2, 3 & +2, +3 \\
\hline
Cobalto & Co & 2, 3 & +2, +3 \\
\hline
Cromo & Cr & 2, 3, 6 & +2, +3, +6 \\
\hline
Manganeso & Mn & 2, 3, 4, 6, 7 & +2, +3, +4, +6, +7 \\
\hline
Titanio & Ti & 2, 3, 4 & +2, +3, +4 \\
\hline
Vanadio & V & 2, 3, 4, 5 & +2, +3, +4, +5 \\
\hline
Platino & Pt & 2, 4 & +2, +4 \\
\hline
Paladio & Pd & 2, 4 & +2, +4 \\
\hline
Molibdeno & Mo & 2, 3, 4, 5, 6 & +2, +3, +4, +5, +6 \\
\hline
Tungsteno & W & 2, 3, 4, 5, 6 & +2, +3, +4, +5, +6 \\
\hline
\caption{Metales con valencias múltiples}
\end{longtable}

\newpage

\section*{No Metales}

\begin{longtable}{|>{\columncolor{nometal}}c|c|c|c|}
\hline
\rowcolor{header}
\textcolor{white}{\textbf{Elemento}} & \textcolor{white}{\textbf{Símbolo}} & \textcolor{white}{\textbf{Valencias}} & \textcolor{white}{\textbf{Estados de Oxidación}} \\
\hline
\endfirsthead

\hline
\rowcolor{header}
\textcolor{white}{\textbf{Elemento}} & \textcolor{white}{\textbf{Símbolo}} & \textcolor{white}{\textbf{Valencias}} & \textcolor{white}{\textbf{Estados de Oxidación}} \\
\hline
\endhead

Hidrógeno & H & 1 & +1, -1 \\
\hline
Oxígeno & O & 2 & -2 (en peróxidos: -1) \\
\hline
Flúor & F & 1 & -1 \\
\hline
Cloro & Cl & 1, 3, 5, 7 & -1, +1, +3, +5, +7 \\
\hline
Bromo & Br & 1, 3, 5, 7 & -1, +1, +3, +5, +7 \\
\hline
Yodo & I & 1, 3, 5, 7 & -1, +1, +3, +5, +7 \\
\hline
Azufre & S & 2, 4, 6 & -2, +2, +4, +6 \\
\hline
Selenio & Se & 2, 4, 6 & -2, +2, +4, +6 \\
\hline
Teluro & Te & 2, 4, 6 & -2, +2, +4, +6 \\
\hline
Nitrógeno & N & 1, 2, 3, 4, 5 & -3, +1, +2, +3, +4, +5 \\
\hline
Fósforo & P & 3, 5 & -3, +3, +5 \\
\hline
Arsénico & As & 3, 5 & -3, +3, +5 \\
\hline
Antimonio & Sb & 3, 5 & -3, +3, +5 \\
\hline
Carbono & C & 2, 4 & -4, +2, +4 \\
\hline
Silicio & Si & 4 & -4, +4 \\
\hline
Boro & B & 3 & +3 \\
\hline
\caption{No metales y sus valencias}
\end{longtable}

\newpage

\section*{Reglas Importantes para Estados de Oxidación}

\subsection*{Reglas Generales}

\begin{enumerate}
    \item El estado de oxidación de un elemento libre (sin combinar) es \textbf{0}.
    
    \item El estado de oxidación del oxígeno es \textbf{-2} en la mayoría de compuestos, excepto:
    \begin{itemize}
        \item En peróxidos: \textbf{-1} (ej: H$_2$O$_2$, Na$_2$O$_2$)
        \item En superóxidos: \textbf{-1/2} (ej: KO$_2$)
        \item Con flúor: \textbf{+2} (ej: OF$_2$)
    \end{itemize}
    
    \item El estado de oxidación del hidrógeno es:
    \begin{itemize}
        \item \textbf{+1} en la mayoría de compuestos
        \item \textbf{-1} en hidruros metálicos (ej: NaH, CaH$_2$)
    \end{itemize}
    
    \item Los metales alcalinos (Grupo 1) siempre tienen estado de oxidación \textbf{+1}.
    
    \item Los metales alcalinotérreos (Grupo 2) siempre tienen estado de oxidación \textbf{+2}.
    
    \item Los halógenos tienen estado de oxidación \textbf{-1} cuando actúan como aniones simples.
    
    \item La suma de los estados de oxidación en una molécula neutra es \textbf{0}.
    
    \item La suma de los estados de oxidación en un ion poliatómico es igual a la \textbf{carga del ion}.
\end{enumerate}

\subsection*{Nomenclatura Tradicional - Sufijos}

\begin{table}[h]
\centering
\begin{tabular}{|c|c|c|}
\hline
\rowcolor{header}
\textcolor{white}{\textbf{Número de Valencias}} & \textcolor{white}{\textbf{Valencia}} & \textcolor{white}{\textbf{Sufijo}} \\
\hline
\multirow{2}{*}{2 valencias} & Menor & -oso \\
\cline{2-3}
 & Mayor & -ico \\
\hline
\multirow{4}{*}{4 valencias} & La menor & hipo-...-oso \\
\cline{2-3}
 & 2ª menor & -oso \\
\cline{2-3}
 & 2ª mayor & -ico \\
\cline{2-3}
 & La mayor & per-...-ico \\
\hline
\end{tabular}
\caption{Sufijos en nomenclatura tradicional}
\end{table}

\subsection*{Ejemplos de Aplicación}

\begin{table}[h]
\centering
\begin{tabular}{|c|c|c|c|}
\hline
\rowcolor{header}
\textcolor{white}{\textbf{Compuesto}} & \textcolor{white}{\textbf{Metal}} & \textcolor{white}{\textbf{E.O.}} & \textcolor{white}{\textbf{Nombre Tradicional}} \\
\hline
FeCl$_2$ & Fe & +2 & Cloruro ferroso \\
\hline
FeCl$_3$ & Fe & +3 & Cloruro férrico \\
\hline
CuO & Cu & +2 & Óxido cúprico \\
\hline
Cu$_2$O & Cu & +1 & Óxido cuproso \\
\hline
HClO & Cl & +1 & Ácido hipocloroso \\
\hline
HClO$_2$ & Cl & +3 & Ácido cloroso \\
\hline
HClO$_3$ & Cl & +5 & Ácido clórico \\
\hline
HClO$_4$ & Cl & +7 & Ácido perclórico \\
\hline
\end{tabular}
\caption{Ejemplos de estados de oxidación}
\end{table}

\newpage

\section*{Tabla Resumen de Valencias más Comunes}

\begin{table}[h]
\centering
\small
\begin{tabular}{|c|c||c|c||c|c|}
\hline
\rowcolor{header}
\textcolor{white}{\textbf{Elemento}} & \textcolor{white}{\textbf{Valencia}} & \textcolor{white}{\textbf{Elemento}} & \textcolor{white}{\textbf{Valencia}} & \textcolor{white}{\textbf{Elemento}} & \textcolor{white}{\textbf{Valencia}} \\
\hline
Li, Na, K & 1 & Al & 3 & Cl, Br, I & 1, 3, 5, 7 \\
\hline
Be, Mg, Ca & 2 & Fe & 2, 3 & S, Se, Te & 2, 4, 6 \\
\hline
Sr, Ba & 2 & Cu & 1, 2 & N & 1, 2, 3, 4, 5 \\
\hline
Zn, Cd & 2 & Ag & 1 & P, As, Sb & 3, 5 \\
\hline
Pb, Sn & 2, 4 & Au & 1, 3 & C & 2, 4 \\
\hline
Cr & 2, 3, 6 & Hg & 1, 2 & B & 3 \\
\hline
Mn & 2, 3, 4, 6, 7 & Ni, Co & 2, 3 & Si & 4 \\
\hline
\end{tabular}
\caption{Resumen rápido de valencias}
\end{table}

\section*{Ion Amonio (NH$_4^+$)}

El ion amonio (NH$_4^+$) actúa como un metal con valencia \textbf{1} en la formación de sales.

\textbf{Ejemplos:}
\begin{itemize}
    \item NH$_4$Cl - Cloruro de amonio
    \item (NH$_4$)$_2$SO$_4$ - Sulfato de amonio
    \item NH$_4$NO$_3$ - Nitrato de amonio
\end{itemize}

\section*{Notas Adicionales}

\begin{itemize}
    \item \textbf{Estado de oxidación} y \textbf{valencia} no son exactamente lo mismo, aunque frecuentemente se usan de manera intercambiable en química básica.
    
    \item El estado de oxidación puede ser positivo, negativo o cero.
    
    \item La valencia generalmente se refiere al número de enlaces que puede formar un átomo.
    
    \item Algunos elementos tienen valencias variables dependiendo del compuesto que formen.
    
    \item En compuestos iónicos, el estado de oxidación corresponde a la carga del ion.
\end{itemize}
